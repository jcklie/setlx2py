%!TEX root = ../main.tex

\section{AST}
\label{sec:ast}

The nodes for the \gls{ast} used  internally are generated from a config file to ease development. For the purpose of this project, the AST-generating part of another open-source compiler project\footnote{\url{https://github.com/eliben/pycparser}}  was forked into another project \footnote{\url{https://github.com/Rentier/ast-gen}} for that. New functionallity was added to ease testing them against expected trees.  Further more, their vizualization was improved.

The idea behind ast-gen is to describe the nodes, their attributes and children in a config file. From it, a Python file containing one class for every node is generated and then can be used without any dependency to the ast-gen project. Therefore, it is only needed in development.

\subsection{Configuration}

Each entry in the config file is a sub-class \texttt{<name>} of Node, listing its attributes and child nodes. Each line contains the name of the class, followed by it attributes:

\verb$<Name>: [list, of, attributes]$

The attributes can be of the following kind:

\begin{verbatim}
<name>*     - a child node
<name>**    - a sequence of child nodes
<name>      - an attribute.
\end{verbatim}

Example:

\begin{verbatim}
BinaryOp: [op, left*, right*]
Constant: [type, value]
ExprList: [exprs**]
\end{verbatim}

The file used in setlx2py can be found in \texttt{setlx2py/setlx_ast.cfg}.

\subsection{Generation}

After installing ast-gen, the Python file can be generated with the folling command in a Python shell:

\begin{lstlisting}{language=python}
from astgen.ast_gen import ASTCodeGenerator
gen = ASTCodeGenerator(**PATH_TO_CONFIG.cfg**)
with open(**PATH_TO_WHERE_TO_SAVE**, 'w') as f:
gen.generate(f)
\end{lstlisting}

Alternatively, the Makefile delivered with setlx2py offers a nice shortcut:

\verb$make ast$