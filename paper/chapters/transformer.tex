%!TEX root = ../main.tex

\section{Transformation}

\label{sec:transformer}

In order to make code generation easier, the parser-generated AST is partially rewritten.  Mostly, that is done because some constructs in SetlX and Python differ quite largely. This maybe could also be done in the parser, but the main reason for a dedicated transformation step is to decouple the components. 

It was tried to avoid altering the parser in order to simplify code generation, as these components should be as simple an as´independent from each other as possible. 

The implementation is similar to the one of code generation: the AST is traversed depth-first, and one can specify actions for every AST node type. 

In this section, the two most interesting and complex cases are described.

%
\subsection{Procedure definitions}

In order to define a function in SetlX, one creates an anonymous function and binds it to a name via assignment.

\begin{lstlisting}[language=c]
signum := procedure(n) {
    if(n > 0) {
        return 1;
    } else if(n < 0) {
        return - 1;
    } else {
        return 0;
    }
};
\end{lstlisting}

In Python, when defining a function, it is automatically bound to that name in the current scope:

\begin{lstlisting}[language=python]
def signum(n):
    if n > 0:
        return 1
    elif n < 0:
        return - 1;
    else:
        return 0;
\end{lstlisting}

The AST for a procedure definition in setlx2py consists of the assignment node itself with name on the left- and the procedure on the right-hand side. As an example, the following procedure is discussed:

\begin{lstlisting}[language=c]
add := procedure(a,b) { return a + b; };
\end{lstlisting}

The corresponding AST is

\begin{lstlisting}[language=lisp]
('Assignment', ':=', 
  ('Identifier', 'add'), 
  ('Procedure', '', 'vanilla', 
    ('ParamList', ('Param', 'a'), ('Param', 'b')), 
    ('Block', 
      ('Return', 
        ('BinaryOp', '+', 
          ('Identifier', 'a'), 
          ('Identifier', 'b'))))))
\end{lstlisting}

One can see that the name of the procedure, (which Python needs for declaring it) is to be found in the assignment node. In the code generation step, only the current node and is children are visible to a callback. That means, the procedure node would have no access to the procedure name. The simple solution is to rewrite the AST for this construct. The rule for that is therefore that if there is an assignment statement where the right-hand side is a procedure, the name is assigned to the procedure node and the assignment node is replaced with its right-hand side:

\begin{lstlisting}[language=python]
def visit_Assignment(self, n, p):
    if isinstance(stmt, Assignment) and isinstance(stmt.right, Procedure):
        assignment = stmt
        procedure = assignment.right
        procedure.name = assignment.target.name
        n.stmts[i] = procedure
\end{lstlisting}

The result of transformation is

\begin{lstlisting}[language=lisp]
('Procedure', 'add', 'vanilla', 
  ('ParamList', ('Param', 'a'), ('Param', 'b')), 
  ('Block', 
    ('Return', 
      ('BinaryOp', '+',
        ('Identifier', 'a'), 
        ('Identifier', 'b')))))
\end{lstlisting}

\todo{Update to rewritten visit\_Assignment}

%
\subsection{Interpolation}

String interpolation in SetlX offers an easy way to inline expressions into strings. It is especially a nice syntax to create strings from data. 

\begin{lstlisting}
s := "signum($n$) is $signum(n)$.";
\end{lstlisting}

As a string is enriched with expressions, these have to be parsed aswell. Therefore, interpolation is only available at compile time. Parsing of the inlined expressions is done after the AST has been generated by the parser. To be more specific, it is done in the transformation phase. The high-level view is very simple:

\begin{lstlisting}[language=python]
def visit_Interpolation(self, n, p):
    self.fill_interpolation(n)
    self.generic_visit(n, p)
\end{lstlisting}

The \token{fill_interpolation} extracts the expressions in the given string, parses them, and creates the string format with the expressions as arguments. The AST after parsing is

\begin{lstlisting}[language=lisp]
('Assignment', ':=', 
  ('Identifier', 's'), 
    ('Interpolation', 
      ('Constant', 'literal', 'signum($n$) is $signum(n)$.'), 
      ('ExprList',)))
\end{lstlisting}

The result of transformation is

\begin{lstlisting}[language=lisp]
('Assignment', ':=', 
  ('Identifier', 's'), 
  ('Interpolation', 
    ('Constant', 'literal', 'signum({0}) is {1}.'), 
    ('ExprList', 
      ('Identifier', 'n'), 
      ('Call', 
        ('Identifier', 'signum'), 
        ('ArgumentList', 
          ('Identifier', 'n'))))))
\end{lstlisting}

Finally, the generated Python code is 

\begin{lstlisting}
s = "signum({0}) is {1}.".format(n, signum(n))
\end{lstlisting}
